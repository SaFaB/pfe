\chapter{Les approches de composition dynamique des services Web sémantiques baseés sur le modèle graphe}
%% Introduction to the chapter
\newpage
\section{Préliminaires}
%% positionner l'approche par graphe dans les classification déjà
%% abordées
  \subsection{Matching des services Web}
  \label{sec:match-des-serv}
  ( \cite{lecue2006formal}, \cite{paolucci2002semantic},
  \cite{li2004software})\\
  Le matching, l'alignement, le mapping...ect. Sont tous des termes
  utilisées pour référencer le concept de recherche de
  similarité. Voici quelques définitions de ces termes utilisés.

  \begin{mydef}[Matching]
    c'est le processus de découverte des liaisons et des
    correspondances entre les entités de différentes représentations à
    travers un algorithme de matching.
  \end{mydef}

  % Matchmaking is defined as a process that requires a repository host
  % to take a query or an advertisement as input, and to return all the
  % advertisements that may satisfy the requirements specified in the
  % input query or advertisement.

  Dans le contexte d'une architecture à services, le
  \textit{Matchmaking} (ou le \textit{Matching}) est définie comme un
  processus qui nécessite un annuaire des services de prendre une
  requête en entrée, et de revenir tous les services qui peuvent
  satisfaire les exigences spécifiées dans la requête d'entrée
  \cite{li2004software}. Formellement, Le processus de
  \textit{Matchmacking} peut se spécifié comme de suit:\\

  Soit $\Phi$ est l'ensemble de toutes les services référencés dans un
  annuaire des services donné. Pour une requête \verb|R|.

  \begin{mydef}[Alignement]
    on parle souvent de l'alignement des ontologies. C'est un ensemble
    de correspondances entre deux ou plusieurs
    ontologies. L'alignement est la sortie d’un processus de matching
  \end{mydef}

  \begin{mydef}[Mapping]
    chercher des correspondances pour établir des transformations
    entre deux objets de même nature mais pas de même forme. Par
    conséquent, le mapping utilise les résultats du matching pour
    effectuer les transformations des objets.
  \end{mydef}

  Comme on le voit, tous ces termes prennent bien en compte la notion
  de recherche de correspondance entre des concepts, dans certains
  contextes ils sont utilisés indifféremment.  Pour les services web,
  on parlera du matchnig ou matchmaking pour exprimer la mise en
  correspondance entre deux entités, et du mapping pour exprimer la
  transformation ou conversion des types de données.

  \subsection{Mésures de similarité}
  \label{sec:mesure-de-similarire}

  \subsection{Graphe de dépendance}
  \label{sec:graph-de-depandence}
  \cite{Omer2011}
\section{Travaux relatives}
\label{sec:travaux-relatives}

% La découverte d’un service composite à partir d’un graphe de
% dépendance consiste à trouver le meilleur chemin dans le graphe qui
% génère les sorties exprimées dans la requête. La recherche peut se
% baser sur l’optimisation d’une fonction d’utilité qui tient compte du
% degré de matching entre les services et aussi de la qualité de service
% qui peut être fournie par les services concrets. A cet effet, les
% méthodes de recherche du meilleur chemin de la théorie des graphes
% peuvent être exploitées.  Les méthodes de composition utilisant le
% modèle de graphe peuvent être classées en deux catégories selon que le
% graphe de dépendance soit construit à priori (en phase de publication
% des services) ou pendant le traitement de la requête de composition :

  \subsection{Génération Online du graphe de dépendance}
  \label{sec:generation-online-du}

  \subsection{Génération Offline du graphe de dépendance}
  \label{sec:gener-offl-du}

  \subsection{Autres travaux basés sur le graphe matching}
  \label{sec:autres-travaux}

\section{Vers les bases de données graphe}
\label{sec:vers-les-bases}

\section{Conclusion}
\label{sec:conclusion-2}


%%% Local Variables:
%%% mode: latex
%%% TeX-master: "../main"
%%% End: