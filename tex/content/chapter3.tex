\chapter{Les approches de composition dynamique des services Web sémantiques baseés sur le modèle graphe}
%% Introduction to the chapter
\newpage
\section{Préliminaires}
\section{Le processus de matching et la découverte des services Web sémantiques}
% positionner l'approche par graphe dans les classification déjà
% abordées


%   \subsection{Mésures de similarité}
%   \label{sec:mesure-de-similarire}
%   objectif des mesures de similarité sémantique est d'évaluer la
%   proximité sémantique entre les concepts (auxquels les termes des
%   requêtes et documents sont rattachés). Et comme le matching des
%   services web revient à trouver les similarités entre les concepts
%   décrivant les propriétés des services web, nous avons jugé utile
%   d’explorer les différentes méthodes de recherche de similarité
%   utilisées dans le domaine de la recherche d’information, en
%   particulier.

%   \begin{mydef}[Similarité entre concepts]
%     Une similarité $\sigma$: C $\times$ C $\rightarrow$ [0,1] est une
%     fonction de couples de concepts vers un nombre compris entre 0 et
%     1 exprimant le degré de similarité entre deux concepts, tel que:
%     \SpecialItemi
%     \begin{itemize}
%       \item x $\in$ C, $\sigma$(x,x) = 1
%       \item x, y $\in$ C, $\sigma$(x,y) = $\sigma$(y,x)
%     \end{itemize}
%   \end{mydef}

%   Les ontologies définissent les propriétés des concepts et les
%   relations entre concepts. Une de ces relations est utilisée dans
%   notre application est \textsc{is–a} (est-un). Cette relation est utilisée
%   pour extraire des taxonomies à partir des ontologies

%   \begin{mydef}[Similarité entre concepts]
%     Soit C l'ensemble de concepts définis dans une ontologie. On dit
%     que T est une taxonomie et on écrit T(C, $\leq$) tel que c $\leq$
%     c' signifie que c \textit{is–a} c' (le concept c est subsumé par
%     le concept).
%   \end{mydef}

%   De nombreuses approches ont été proposées pour évaluer la similarité
%   sémantique entre deux concepts. Ces approches se divisent en trois
%   catégories [Sli, 2006]: les approches basées sur les arcs, les
%   approches basées sur le contenu informationnel et les approches
%   hybrides. %% see annexe (1)

% \section{Matching des services Web sémantiques }
% \label{sec:match-des-serv-1}
% Le processus de matchmaking repose sur la recherche de similarités
% entre les paramètres descriptives de ces derniers. Il existe deux
% catégories de paramètres de description des services web: les
% paramètres fonctionnels et les paramètres non-fonctionnels que nous
% décrivons tout de suite.

%   \subsection{Les paramètres fonctionnels}
%   \label{sec:params-fonc}

%   La sémantique fonctionnelle décrit les fonctionnalités qu’offre le
%   service en terme de transformation d’information dénotée par les
%   Entrés/Sorties (I/O) et de changement d’état après l’exécution du
%   service dénoté par les pré-conditions/effets (ou post-conditions)
%   (P/E).  Ils sont appelés ainsi parce qu’ils sont nécessaires pour le
%   fonctionnement du service web. Ces paramètres sont utiles dans la
%   recherche des similarités. En effet, grâce au formalisme de
%   description des services web (OWL-S), on trouve une description des
%   données requises pour l’exécution du service (Inputs) et une
%   description des résultats obtenus après exécution (Outputs). D’un
%   autre coté, un service web peut altérer l’état du monde après son
%   exécution.  L’état du monde requis pour que le service s’exécute est
%   la pré-condition et le nouvel état généré après son exécution est
%   l’effet du service sur le monde. Par exemple le service de logging à
%   un site web possède comme information d’entrées le username et le
%   password et l’information de sortie est un message de
%   confirmation. Après l’exécution, l’état du monde change de
%   not\_logged\_in à logged\_in. Prenons un autre exemple, un transfert
%   d’argent du compte A vers le compte B ne peut se faire que si le
%   compte A est solvable (les pré-conditions).  Les effets décrivent
%   les sorties (Output) et l’impact d’une exécution d’un service
%   web. Tant dis que le terme post-condition focalise souvent sur la
%   description des conditions concernant les valeurs retournées des
%   opérations du service web. Dans notre exemple ci-dessus, un impact
%   de l'exécution serait le transfert réel de l'argent du compte de
%   carte de crédit pour le compte de destination alors que la
%   post-condition décrit qu'aucun argent ne s’est perdu lors du
%   transfert [Car, 2007].  L’opération de matching consiste alors à
%   relier les paramètres des différents services (faire correspondre)
%   entre eux et cela suivant des règles de correspondance.

%   \subsection{Les paramètres non-fonctionnels}
%   \label{sec:params-non-fonc}

%   Les paramètres non fonctionnels sont des paramètres qui ne sont pas
%   reliés directement aux fonctionnalités d’un service web, mais mesure
%   la qualité avec laquelle le service délivre ses
%   fonctionnalités. Parmi ces paramètres on trouve:

%   \SpecialItem
%   \begin{description}
%   \item [Le coût]: C’est une propriété non-fonctionnelle qui est
%     pertinente pour le client qui veut utiliser le service.

%   \item [La sécurité]: La sécurité est une propriété non-fonctionnelle
%     qui est pertinente à la plupart des services web. Elle applique
%     des aspects tels que la communication et le cryptage des données.

%   \item [La qualité]: La propriété qualité des services est un
%     ensemble de propriétés différentes qui affecte tous les aspects
%     spécifiques du service, son utilisation et sa production tel que
%     le temps de réponse d'un appel à une opération, la capacité du
%     service...etc
%   \end{description}

%   Le matching des paramètres non-fonctionnels des services web est
%   tout aussi important dans la phase de sélections des services
%   candidats. Les propriétés non-fonctionnelles sont typiquement
%   représentés comme des politiques de services web exprimées par le
%   langage WS- Policy [Car, 2007].

%   Généralement ces approches de matching des politiques peuvent être
%   utilisées pour l'optimisation et l’évaluation des plans de
%   composition.  Les services web correspondent à des entités
%   dynamiques. Ils peuvent devenir disponibles à tout moment de
%   l’exécution du système et découverts dynamiquement par leurs
%   clients. Les fournisseurs et les consommateurs de services doivent
%   donc disposer d’un moyen commun et fiable pour effectuer la
%   publication et la recherche de services. On entend par découverte
%   dynamique la possibilité de localiser automatiquement un ensemble de
%   services web, sélectionner un service spécifique qui répond à des
%   besoins particuliers de l’utilisateur.  Plusieurs initiatives ont
%   traité la problématique de la découverte des services web dont les
%   deux principales modèles suivants:

%   \subsection{Le modèle syntaxique de matching des services Web}
%   \label{sec:le-modele-syntaxique}

%   \subsection{Le modèle sémantique de matching des services Web}
%   \label{sec:le-modele-semantique}

%   \subsection{Graphe de dépendance}
%   \label{sec:graph-de-depandence}
%   \cite{Omer2011}

\section{Travaux connexes}
\label{sec:travaux-relatives}

% La découverte d’un service composite à partir d’un graphe de
% dépendance consiste à trouver le meilleur chemin dans le graphe qui
% génère les sorties exprimées dans la requête. La recherche peut se
% baser sur l’optimisation d’une fonction d’utilité qui tient compte du
% degré de matching entre les services et aussi de la qualité de service
% qui peut être fournie par les services concrets. A cet effet, les
% méthodes de recherche du meilleur chemin de la théorie des graphes
% peuvent être exploitées.  Les méthodes de composition utilisant le
% modèle de graphe peuvent être classées en deux catégories selon que le
% graphe de dépendance soit construit à priori (en phase de publication
% des services) ou pendant le traitement de la requête de composition :

  \subsection{Génération Online du graphe de dépendance}
  \label{sec:generation-online-du}

  \subsection{Génération Offline du graphe de dépendance}
  \label{sec:gener-offl-du}

  \subsection{Autres travaux basés sur le graphe matching}
  \label{sec:autres-travaux}

\section{Vers les bases de données graphe}
\label{sec:vers-les-bases}

\section{Conclusion}
\label{sec:conclusion-2}


%%% Local Variables:
%%% mode: latex
%%% TeX-master: "../main"
%%% End: