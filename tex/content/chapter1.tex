%!TEX root = ../main.tex
% < 20 pages

% \chapter{Concepts de base du paradigme service Web}
% Une introduction implicite du chapitre %TODO
    % Dans ce chapitre on va faire une vue générale sur les technologies des services web sémantiques, Pour bien 
    % introduire le chapitre prochain sur la composition des services web.

    Ce chapitre établit une étude du fondement théorique de notre travail à savoir les concepts de base du paradigme
    service Web.  Nous commençons d'abord par présenter un tour d'horizon définissant l'infrastructure et
    l'architecture de référence de ce paradigme ainsi que quelque définitions proposées par la littérature. Ensuite
    nous nous intéressons à montrer les limitation de l'approche syntaxique de la description des services web et l'apport
    de l'enrichissement sémantique de cette dernière aux processus de la découverte et la composition des services Web.\\
    % TODO: complete the introduction

\section{Les services web : Notions de base et technologies associées} 
    Les services Web constituent une approche pour mettre en œuvre le paradigme de service, et peut être vue comme
    une instance de l'architecture orienté service.

    Dans cette section va parler aussi d'un socle technologique très sollicité, On va aussi Détailler l'architecture de base 
    d'un service web, ensuite nous introduisons l'architecture étendus.

    \subsection{Définitions et caractéristiques } 
	Les services Web sont la technologie la plus connue et la plus populaire dans le monde industriel et
	académique pour la mise en place d’architectures à services. 
	% ARE PROGRAMMABLE COMPONENTS which use the World-
	% Wide-Web as a medium for describing the functionality of real world ser-
	% vices in a computer manipulable way (Kuropka and Nern, 2006)
	Les Web services ont été proposés initialement par IBM \cite{kreger2001web} et Microsoft, puis en partie 
	standardisés par le \acrshort{w3c} \footnote{\url{http://www.w3.org/}} et définis \cite{WSA} par:\\

	\emph{``Un service web est un système conçu pour permettre d'interopérabilité des applications à travers un réseau. 
	    Il est caractérisé par un format de description interprétable/compréhensible automatiquement par la machine,
	    D'autres systèmes peuvent interagir avec le Service Web selon la manière prescrite dans sa description et en
	    utilisant des messages SOAP, généralement transmis vie le protocole HTTP et sérialisés en XML et en d'autres 
	    standards du Web ''}.\\

	Cette définition surligne les caractéristiques clés de services Web \cite{fremantle2002enterprise}:

	\SpecialItem
	\begin{description} % Les caractéristiques des services web Selon la définition de W3C
	    \item[Basés sur des protocoles Internet] : 
		L'utilisation de \acrshort{http} pour le transport des informations permet de traverser les contrôles
		d'accès dans un environnement hétérogène.

	    \item[Interopérables] : 
		Le standard \textsc{SOAP} \cite{box2000simple} définit comme étant un protocole destiné à l'échange 
		de messages structurés véhiculé généralement sur \acrshort{http} et sérialisé en \textsc{XML}, 
		permettant le support pour l'interopérabilité.

	    \item[Basés sur XML] : 
		Le méta-langage de balisage \textsc{XML} \textit{eXtensible Markup Language} est un standard Web ouvert 
		par \textsc{W3C} \cite{bray1998extensible} offre un cadre standard pour la définition de documents 
		Interprétable par des machines.
	\end{description}

	M. P. Papazoglou \cite{papazoglou2003service} apporte une autre définition de services web:\\

	\emph{``Les services Web sont des éléments auto-descriptifs et indépendants des plateformes qui permettent 
	    la composition faible coût d’applications distribuées. Les services Web effectuent des fonctions allant 
	    de simples requêtes des processus métiers complexes. Les services Web permettent aux organisations d’exposer 
	    leurs programmes résultats sur Internet (ou sur un intranet) en utilisant des langages (basés sur XML)
	    et des protocoles standardisés et de les mettre en œuvre via une interface auto-descriptive basée sur 
	    des formats standardisés et ouverts''}\\
	% TODO make a comment on this def and introduce the web services composition idea !

	 
	% TODO organize this mess
	% * Permettre l'interopéabilité des applications, indépendamment des plates-formes et des langages

	% * Permettre le couplage faible des applications (évolution indépendante) et leur coopération 
	% via des interfaces de haut niveau d'abstraction (services globaux)

	% * Permettre une coopération des applications avec un minimum d'intervention humaine
	Curbera et al. \cite{curbera2001web} de ça part proposent la définition suivante :\\

	\emph{``Un service Web est une application réseau capable d'interagir par le moyen des standards et des protocoles 
	via des interfaces bien spécifiés, dans lequel est décris utilisant un langage de description fonctionnel
	standardisé''} .\\

	% Brièvement, un service Web est une entité logiciel modulaire, auto-descriptive et autonome accessible 
	% via Internet.

    \subsection{L'évolution des styles des services web}
    % Les architecture communes des services web
	% SOAP vs REST
	% les limites de la'approche SOAP 
	% pourquoi SOAP ?
	``the next section provides a short history of web services, with emphasis on the kinds of software challenges
	that web services are meant to address.''

    \subsection{L'architecture de référence et technlogies associées}
	\cite{curbera2002unraveling} \cite{gottschalk2002introduction} \cite{kreger2001web} \cite{WSA}
	% The basic SOAP/WSDL/UDDI standards are a particular implementation of the concept of a service-
	% oriented architecture.
	% cette sous-section va détaillé SOAP et pas de WSDL et UDDI
	Cette architecture a été proposée afin de promouvoir l’interopérabilité et l’extensibilité des services Web

	Dans l'ensemble, une architecture complète de services Web est constitué d'un fournisseur de servicee
	\footnote{Providers}, un annuaire de services \footnote{Service Registry}, et un client 
	\footnote{Service Requester} de service.  La figure x montre comment ces trois rôles interagissent.\\

	\SpecialItem
	\begin{description} % TODO tranlate: Les Rôles dans une architecture WSA
	    \item[Le fournisseur]
		A service provider provides the interface for the Web service and the implementation of the
		application. The service provider is also responsible for creating the definition of the service and
		publishing that definition to meet the Universal Description, Discovery, and Integration (UDDI)
		specification.

	    \item[L'annuaire]
		A service registry is a way in which Web services are formally published. The service registry is based
		on the UDDI specification and reflects information about services provided by the service provider. The
		service registry provides a service requester with a Web Services Description Language (WSDL) service
		description and a Uniform Resource Locator (URL) that points to the service itself.

	    \item[Le client]
		A service requester is the consumer of a Web service and uses the service registry to gain information
		about, and access to, a Web service.

	\end{description}

	For an application to take advantage of Web Services, three behaviors must take place:
	publication of service descriptions, lookup or finding of service descriptions, and binding or
	invoking of services based on the service description. These behaviors can occur singly or
	iteratively. In detail, these operations are:
	\SpecialItem 
	\begin{description}% TODO tranlate: WS actions
	    \item[Publish]
		To be accessible, a service description needs to be published so that the
		service requestor can find it. Where it is published can vary depending upon the
		requirements of the application (see “Service Publication” for more details).

	    \item[Find]
		In the find operation, the service requestor retrieves a service description directly
		or queries the service registry for the type of service required (see “Service Discovery”
		for more details). The find operation can be involved in two different lifecycle phases for
		the service requestor: at design time to retrieve the service’s interface description for
		program development, and at runtime to retrieve the service’s binding and location
		description for invocation.

	    \item[Bind]
		Eventually, a service needs to be invoked. In the bind operation the service requestor invokes 
		or initiates an interaction with the service at runtime using the binding details in the service 
		description to locate, contact and invoke the service.  
	\end{description}

	% \begin{description} % SOAP + WSDL + UDDI
	Les services Web sont construits autour de standards qui sont \acrshort{soap}, \acrshort{wsdl} et \acrshort{uddi} 
	assurant respectivement leur communication, leur description et leur découverte .

	\subsubsection{Communication: SOAP}
		\textsc{SOAP}, développé par IBM\footnote{\url{http://www.ibm.com}} et
		Microsoft\footnote{\url{http://www.microsoft.com}} \cite{box2000simple}, est une recommandation \textsc{W3C} 
		qui le définit comme étant un protocole destiné à l'échange de messages structurés, permettant 
		d'invoquer des applications sur des réseaux distribués \cite{mitra2003soap}.

		Ce protocole \textsc{SOAP} est basé sur \textsc{XML} pour mettre en place un mécanisme valable
		d'échange des données indépendant du modèle de programmation de l'application et du système
		d’exploitation.

	      	Un message \textsc{SOAP} est un document XML constitué d'une enveloppe \textsc{SOAP} obligatoire, 
		d'un en-tête \textsc{SOAP} facultatif et d'un corps \textsc{SOAP} obligatoire:

		% TODO pick something real from the app.
		% go to the annexes part
		% \begin{figure}[h]
    \begin{Verbatim}[frame=single, fontsize=\scriptsize]
	<?xml version="1.0" encoding="utf-8"?>
	<soapenv:Envelope
	xmlns:soapenv="http://schemas.xmlsoap.org/soap/envelope/"
	xmlns:xsd="http://www.w3.org/2001/XMLSchema"
	xmlns:xsi="http://www.w3.org/2001/XMLSchema-instance">
	<soapenv:Body>
	<symbol xmlns="http://stock.samples">XXX</symbol>
	</soapenv:Body>
	</soapenv:Envelope>
    \end{Verbatim}
    \caption{Exemple de message SOAP}
    \label{fig:soap-message-example}
\end{figure}


		% TODO SOAP message structure
		% \begin{figure}
		%     \centering
		% \end{figure}

		\SpecialItem
		\begin{description} % SOAP components
		    \item[Enveloppe]:  
			L'élément racine du message \textsc{SOAP} , définissant le contexte du message, son
		       destinataire et son contenu, il englobe l'en-tête et le corps.

		    \item[En-tête \texttt{<Header>}]:
		       	Un mécanisme générique permettant d'ajouter des fonctions à un message
			\textsc{SOAP} d'une manière modulaire sans accord préalable entre les parties en communication. 
			Des exemples d'extension qui peuvent être implémentées comme des en-têtes sont des 
			authentifications, des transactions, des paiements

		    \item[Corps \texttt{<Body>}]: 
			Contient les informations obligatoires destinées à l'ultime destinataire du message, il sert 
			comme un container pour les informations mandataires à l'intention du récepteur du message.
			\textsc{SOAP} définit un élément pour le corps, qui est l'élément \texttt{<Fault>} (Erreur) 
			utilisé pour rapporter les erreurs.
		\end{description}

	\subsubsection{Description: WSDL} %TODO rewrite
	Le langage de description des services Web \acrshort{wsdl} \cite{chinnici2007web} est une recommandation du
	\acrshort{w3c}, maintenant dans sa deuxième version.  \textsc{WSDL} est basé sur \textsc{XML}
	pour décrire les fonctions opérationnelles de services Web. La description des\textsc{WSDL} sont 
	composées d'un interface et des implémentations. L'interface est une définition abstraite et 
	réutilisable service qui peut être référencée par plusieurs implémentations. 	
	% TODO WSDL 2.0

	Le WSDL sert à décrire :
	% TODO
	\renewcommand{\labelitemi}{$\bullet$}
	\begin{itemize} % ce que WSDL offrire
	    \item le protocole de communication (SOAP RPC ou SOAP orienté message)
	    \item le format de messages requis pour communiquer avec ce service
	    \item les méthodes que le client peut invoquer
	    \item la localisation du service.
	\end{itemize}
        % TODO reference the section
	\subsubsection{Découverte: UDDI}
	\acrshort{uddi} \cite{clement2004uddi} est une standardisation pour la publication et la découverte 
	des services Web initialement conçue et spécifiée par le Consortium de standards OASIS\footnote{\url{https://www.oasis-open.org}},
	et il est le résultat d'un accord d'un ensemble d’industriels Ariba\footnote{\url{http://www.ariba.com/}}, 
	IBM, Microsoft, etc en vue de devenir le registre standard de la technologie des services Web.

	\textsc{UDDI} complète les technologies basiques de services Web en permettant de créer un \textbf{annuaire} 
	permettant de localiser sur le réseau le services web recherchés, les services référencés dans \textsc{UDDI} 
	sont accessibles par l'intermédiaire du protocole de communication \textsc{SOAP}, et la publication des 
	informations concernant les fournisseurs et les services doit être spécifiée en \textsc{XML} afin que la 
	recherche et l'utilisation soient faites de manière \textbf{dynamique} et \textbf{automatique}.

	Un \textsc{UDDI} peut appartenir à un domaine public comme internet ou tout autre réseau accessible à un nombre
       	non limité d’utilisateurs, comme il peut appartenir à un domaine restreint comme l'intranet d’une entreprise 
	ou d'un groupe d'entreprise.
	%TODO UDDI discovery and binding example

	Les données stockés dans l'UDDI sont structurées (en \textsc{XML}) et organisées en trois parties 
	connues:

	% TODO
	\begin{description} % les pages d'UDDI
	    \item[Pages blanches]:
		fournissent des descriptions générales sur les fournisseurs de services à savoir le nom de 
		l'entreprise qui fournit le service, son identificateur commercial, ses adresses, etc.

	    \item[Pages jaunes]:
		comportent des descriptions détaillées sur les fournisseurs de services catalogués dans les pages 
		blanches d'une de façon taxonomique (selon secteurs d'activités par exemple).

	    \item[Pages vertes]:
		fournissent des informations techniques sur les services Web catalogués. Ces informations incluent 
		la description du service, les adresses \textsc{URL}, du processus de son utilisation 
		et des protocoles utilisés pour son invocation.

	\end{description}
	% TODO talks about OWL-S and why UDDI approach is semantically poor

     % TODO Conclure cette subsection par la mention du problème de la découverte automatique des services Web et l'insuffisance
     % de la description syntaxique.


\section{Description des services web} 
    % \cite{sivashanmugam2003adding}
    % \cite{mcilraith2003bringing}
    % \cite{vargas2009challenges}
    % \cite{mcilraith2001semantic}
    % \cite{medjahed2004thesis}

    % TODO Refactore and eliminate some repetition 
    % TODO  get more from \¢ite{lopez2008selection}
    Une description du service Web est un document par lequel le fournisseur de services communique au client 
    les spécifications pour invoquer le service Web, Dans cette section nous présentons les modèles de description
    des services web. Nous détaillons dans la première sous-section le modèle de description syntaxique \acrshort{wsdl}
    \cite{chinnici2007web} développé et standardisé par le \acrshort{w3c} qui est devenu un élément essentiel 
    dans des technologies services web. Ensuite en mettant l'accent sur les limitations majeurs de cette
    approche dans un environnement hétérogène qui nécessite une certaine degré de dynamisité et d'automatisation.
    Finalement, Nous présentons les divers approches sémantiques visant à préciser la description d'un
    service en insistant sur les approches d'annotation sémantique et sur les ontologies de services.

    	\subsection{Description syntaxique de services}
	% TODO exoend this introduction and give some examples form a real senario
	 Le langage de description de services Web \acrshort{wsdl} \cite{chinnici2007web} fournit un modèle ainsi
	 qu'un langage basé sur \textsc{XML} de description de services Web.  Un fichier \textsc{WSDL} comprend une 
	 description des fonctionnalités d'un service, mais il ne se préoccupe pas de l'implantation de celles-ci. 
	 Il contient aussi des informations concernant la localisation du service, ainsi que les données et les 
	 protocoles à utiliser pour l'invoquer. En pratique, le document \textsc{WSDL}
	 \footnote{\url{http://www.w3.org/TR/wsdl20/}} est un document \textsc{XML} qui se divise en 
	 deux parties \cite{elie2010}:


       	 \SpecialItem 
	 \begin{itemize}  % WSDL two main parts
	    \item La définition \textbf{abstraite} de l'interface du service avec les opérations supportées par
		le service Web, ainsi que leurs paramètres et les types des données.

	    \item La définition \textbf{concrète} de l'accès au service avec la localisation, par une adresse
		réseau du fournisseur de service \footnote{Service Endpoint}, et les protocoles spécifiques d'accès.
	 \end{itemize}

	% TODO exemple pratique d'un document WSDL associé avec le document SOAP de la section précédente
	Un document \textsc{WSDL} constitué de quatre éléments principaux \cite{chinnici2007web}: 
	\texttt{<Types>}, \texttt{<Interface>}, \texttt{<Binding>}, \texttt{<Service>}.

	 \cite{baryannis2010}
	 \cite{elie2010}
	 \cite{lopez2008selection}
	\renewcommand{\descriptionlabel}[1]{\hspace{1.5cm}\texttt{#1}} 
	\begin{description} % WSDL20 elements

	    \item[<Types>]: 
		L'élément \texttt{Types} sert à un conteneur définissant les données figurant dans les messages 
		échangés par le service. \textsc{WSDL} supporte des types élémentaires prédéfinis (tels que les entiers,
	       	les chaînes de caractères et les dates). Si les données échangées possèdent une structure particulière, 
		il est possible de les décrire à travers un schéma XML \cite{part20012}.
		
	    \item[<Interface>]:
		Les interfaces\textsc{WDSL} offrent une manière abstraite de décrire la fonctionnalité du service,
		Contrairement à la représentation concrète offerte par les éléments de \texttt{<Bindings>} et 
		de \texttt{<Services>} qui sera décrit plus tard. 
		Une interface \texttt{WSDL} est constitué d'un ensemble d'opérations, chacun d'entre eux décrivant 
		d'une simple interaction entre le service et le client. Une opération décrit un séquence des messages
		d'entrées/sorties ou un modèle d'échange de message \footnote{message exchange pattern} suivie lorsque
		l'opération est invoqué. Pour chaque message contenu dans le motif \footnote{pattern},
	       	un type de message est spécifié à l'aide des types qui ont été définis précédemment dans le document. 
		\textsc{WSDL} contient huit modèles de messages prédéfinis, mais on peut facilement définir de nouveaux.
	    %TODO refactor this part
	    \item[<Binding>]: \cite{elie2010} \cite{baryannis2010}
		L'élément \texttt{Binding} reprend les opérations de l'élément \texttt{<Interface>} et leurs associe 
		un protocole de transfert et des spécifications des formats de données de message.  
		La définition des protocoles de communication utilisés pour l'invocation du service Web permet 
		d'établir le lien, d'une part, entre le document et les messages \textsc{SOAP} et d'autre part,
	       	entre les messages \textsc{SOAP} et les opérations invoquées.  

		% A WSDL binding specifies concrete message format and transmission protocol details
		% for an interface. Each operation in a WSDL description must be associated with a
		% binding. Several typical binding extensions are defined in WSDL 2.0 such as SOAP
		% binding or HTML binding, but the service provider is free to use others provided that
		% they are supported by the client and the service toolkits. The syntax for bindings
		% parallels the syntax of interfaces, since each interface construct has a binding counter-
		% part. A binding can either be reusable, meaning that it is applicable to any interface,
		% or not.

	    \item[<Service>]:
		Cet élément définit la localisation du service Web décrit. Pour chaque
		interface décrite, un élément service lui est associé. Le sous-élément \texttt{<endpoint>} définit un
		port d’accès en référençant l’élément \texttt{<binding>} associé et en déclarant l'\textsc{URL} 
		localisant le service (avec l’attribut \texttt{<address>}). 	    	

		% A WSDL service element specifies a single interface that the service will support, and
		% a list of endpoint locations where that service can be accessed. Each endpoint must
		% also reference a previously defined binding to indicate what protocols and transmis-
		% sion formats are to be used at that endpoint. A service is only permitted to have
		% one interface but there are workarounds for a service to be able to expose multiple

		% précise l’adresse ou les adresses aux-quelles se trouve le service. Un <service> est un ensemble 
		% de <port>. Un <port> spécifie une adresse pour un <binding> donné. Par exemple

		% Cet élément définit la localisation du service Web décrit. Pour chaque
		% interface décrite, un élément service lui est associé. Le sous-élément endpoint définit un
		% port d’accès en référençant l’élément binding associé et en déclarant l’URL localisant le
		% service (avec l’attribut address). Ceci permet qu’une interface d’un service possède plus d’une
		% localisation (i.e. plus d’un élément endpoint) pour répondre aux problèmes d’indisponibilité.
	\end{description}

        \subsection{Ajout de la sémantique}
	    % Pour pallier le manque de sémantique de
	    % WSDL, plusieurs approches proposent de rajouter une couche au dessus de WSDL
	    % complétant la description syntaxique par des précisions sémantiques.
	    % \cite{lopez2008selection}
	    Malgré les améliorations apportées au standard \textsc{WSDL} dans son deuxième version
	    \cite{chinnici2007web}, la description du service reste uniquement au niveau fonctionnel, 
	    c'est-à-dire qu'elle contient la manière dont on peut utiliser le service et non ce
	    que fait le service, le standard \textsc{WSDL} est limité à l'énumération des opérations et 
	    à la description des types des paramètres d'entrée et de sortie associés, elle ne caractérise 
	    pas la sémantique de la fonctionnalité accomplie par le service. Par conséquent, la description
	    \textsc{WSDL} reste insuffisante lors du processus de sélection. 
	    Pour pallier cette Difficulté, plusieurs approches proposent de rajouter 
	    une couche au dessus sémantique de \textsc{WSDL} complétant la description syntaxique par 
	    des précisions sémantiques.\\
	    % explain the problem in the context if web services composition  using ...
	    % \cite{bartalos2011effective}

	    Dans un premier temps, on va essayer de clarifier la notion d'un services Web sémantique,
	    puis étudie les langages émergeants qui permettent de décrire ce type de services Web.
	    
	    \subsubsection{Définition des services Web sémantiques} 
	    % TODO the semantic web stack in the appendix 

	    % la description syntaxique est insuffisante.
	    % A Semantic Web service is defined as an extension of Web service description through the Semantic Web annotations,
	    % created in order to facilitate the automation of service interactions . Therefore, from 
	    % he perspective of the functionality offered, Semantic Web services are still Web services. The only difference lays
	    % in their description and the consequent benefits that follow, namely the reduction of human involvement in 
	    % he performed interactions.\\

	    % What is semantic web? 
	    
	    % Le Web tel que nous le connaissons aujourd'hui est encore conforme à la vision initial 

	    % Le Web a été conçu principalement pour une utilisation par les humains. Néanmoins, il existe un effort
	    % visant à automatiser son utilisation et pour apporter le Web plus accessible pour les machines

	    % \cite{bartalos2011effective}
	    % The Web was primarily designed for use by humans. Nevertheless, there is an ef-
	    % fort to automate its use and bring the Web more accessible for machines. This has
	    % brought forward the need for machine processable representations of semantically
	    % rich information. This has brought forward the need for machine processable representations
	    % of semantically rich information: a vision at the heart of the Semantic Web

	    L'objectif premier du Web sémantique est de définir et lier les ressources du Web afin de simplifier
	    leur utilisation, leur découverte, leur intégration et leur réutilisation dans le plus grand nombre
	    d'applications \cite{berners2001semantic}. Le Web sémantique doit fournir l'accès à ces ressources par
	    l'intermédiaire de descriptions sémantiques exploitables et compréhensibles par des machines. En effet,
	    Les technologies du Web sémantique complètent le Web actuel avec des outils sémantiques. Il ne s'agit donc
	    pas de créer un nouveau Web ou un Web séparé de l'existant : ce Web de données repose entièrement sur
	    les technologies et concepts qui ont fait le succès du Web tel que nous le connaissons aujourd'hui
	    \cite{bertails2010web}.

	    La réalisation du Web sémantique trouve ces racines dans le développement des langages de balisage inspiré
	    par des travaux issue de la communié AI \cite{mcilraith2001semantic}, tels que 
	    \textsc{OIL} \cite{fensel2001oil}, \textsc{DAML+OIL} \cite{horrocks2002daml+oil} et \textsc{DAML+OTN} 
	    \cite{mcguinness2003daml} (ces deux derniers langages sont parties de la famille \acrshort{daml}).

	    Ces langaes ont une sémantique bien définies et permettent le balisage et la manipulation des taxonomique
	    complexe et Des relations logiques entre les entités sur le Web.

	    % \cite{lopez2008selection}
	    Cette description repose sur des ontologies. Selon [Gruber, 1993], une ontologie est une spécification
	    explicite d'une conceptualisation. Une conceptualisation est un modèle abstrait qui représente la
	    manière dont les personnes conçoivent les choses réelles dans le monde et une spécification explicite
	    signifie que les concepts et les relations d'un modèle abstrait reçoivent des noms et des définitions
	    explicites [Gruber, 1993]. Le Web sémantique est devenu un domaine à part entière, preuve en est la
	    création en 2001 du groupe de travail sur ce sujet par le \textsc{W3C}.

	    %  % Semantic web services
	    % \cite{fensel2001oil}

	    % What I need here ?
	    % the base 
	    % \cite{lopez2008selection}

	    % \cite{berners2001semantic} % semantic web
	    % \cite{bray1998extensible} % XML
	    % \cite{part20012} % XML schema
	    % \cite{lassila1999resource} %RDF
	    % \cite{brickley2000resource} % RDFS
	    % \cite{mcguinness2004owl} % OWL
	    % \cite{decker2000semantic} % The semantic web: The roles of XML and RDF
	    % \cite{gruber1993translation} % get ontologies def
	    % \cite{uschold1996ontologies}
	    % \cite{fensel2002web}  % service semantic web between semantic web and the web services
	    % \cite{mcilraith2003bringing}  % Bringing semantics to web services
	    % \cite{sivashanmugam2003adding} % Adding Semantics to Web Services Standards.

	    %TODO define the nedeed for the semantic
	    % an example here will be nice !
	    % - insuffisance de description syntaxique des services web :(WSDL)
	    \subsubsection{Annotations sémantiques}
	    L'annotation sémantique consiste à enrichir et à compléter la description d'un service\cite{elie2010}. 
	    Elle établit des correspondances entre des éléments de la description et des concepts d'un ensemble 
	    d'ontologies de référence. Une ontologie de référence permet de représenter un domaine par des structures 
	    interprétables par une machine. Deux modèles principaux suivent l'approche d'annotation sémantique, 
	    à savoir \textsc{WSDL-S}\cite{akkiraju2005web}, \acrshort{sawsdl} \cite{kopecky2007sawsdl}:
	    % et \textsc{METEOR-S} \cite{patil2004meteor}.

	    \renewcommand{\descriptionlabel}[1]{\hspace{1cm}\textbf{#1}} 
	    \begin{description}
		\item[WSDL-S]: 
		    C'est le résultat d'un travail collaboratif entre IBM et au laboratoire LSDSI. La spécification 
		    a devenue une recommandation \textsc{W3C} depuis 2005. Son objectif principal est de fournir un
		    processus d'annotation sémantique compatible avec les technologies existantes. Pratiquement, Le
		    méta-modèle \textsc{WSDL-S} repose sur les capabilités du modèle \textsc{WSDL} 
		    en rajoutant trois éléments majeurs \texttt{<category>}, \texttt{<effect>} et deux deux attributs
		    \texttt{modelReference} et \texttt{schemaMapping}. Les éléments introduits permettent de 
		    rajouter des informations qui n'étaient pas prises en compte dans \texttt{WSDL} comme 
		    \emph{les préconditions} et \emph{les effets} d'une opération. Tandis que les attributs permettent 
		    de référencer des concepts dans des ontologies de référence, ces préconditions et effets ensemble
		    avec les annotations sémantiques des éléments \texttt{<inputs>} et \texttt{<outputs>} permet de
		    l'automatisation du processus de découvert de services.

		    % TODO example of wsdl-s documeent \cite{elie2010}

		    % The WSDL [WSDL] document forms the anchor point for Web services description. Building on the
		    % descriptive capability of WSDL, we provide a mechanism to annotate the capabilities and
		    % requirements of Web services with semantic concepts referenced from a semantic model. To do this,
		    % we provide mechanisms to annotate the service and its inputs, outputs and operations. Additionally,
		    % we provide mechanisms to specify and annotate preconditions and effects of Web Services. These
		    % preconditions and effects together with the semantic annotations of inputs and outputs can enable
		    % automation of the process of service discovery.
		   % TODO complete this list wit a real uses case with a sample wsdl-s document
		    \begin{itemize}
			\item L'élément \texttt{<category>}
			\item \texttt{<precondition>}
			\item \texttt{<effect>}
			\item L'attribut \texttt{modelReference}
			\item \texttt{schemaMapping}
		    \end{itemize}
		\item[SAWSDL]:
	    \end{description}
	    \subsubsection{Ontologies de services} 
	    Une ontologie de services saisit les différents aspects liés à la description des services et leur 
	    utilisation à travers un ensemble de concepts, de propriétés et de relations entre eux. Tois modèles
	    d'ontologies de services sont décrits ci-après \textsc{OWL-S} \cite{martin2004owl}, \acrshort{wsmo} 
	    \cite{de2006web} et \acrshort{swsf} \cite{battle2005semantic}:

	    \begin{description}
		\item[OWL-S]
		    \cite{mcilraith2003bringing}
		\item[WSMO ]
		\item[SWSF ]
	    \end{description}

\section{Découverte des services web}
   % parler de l' UDDI,le matching sémantique !!

\section{Conclusion}
    % faire un petit récapitulatif sur les technologies des services web 
    % rappeler de notre problème principale : composition des services web
