%!TEX root = ../main.tex
% < 20 pages

% \chapter{Concepts de base du paradigme service Web}
% Une introduction implicite du chapitre %TODO
    % Dans ce chapitre on va faire une vue générale sur les technologies des services web sémantiques, Pour bien 
    % introduire le chapitre prochain sur la composition des services web.

    Ce chapitre établit une étude du fondement théorique de notre travail à savoir les concepts de base du paradigme
    service Web.  Nous commençons d'abord par présenter un tour d'horizon définissant l'infrastructure et
    l'architecture de référence de ce paradigme ainsi que quelque définitions proposées par la littérature. Ensuite
    nous nous intéressons à montrer les limitation de l'approche syntaxique de la description des services web et l'apport
    de l'enrichissement sémantique de cette dernière aux processus de la découverte et la composition des services Web.\\
    % TODO: complete the introduction

\section{Les services web : Notions de base et technologies associées} 
    Les services Web constituent une approche pour mettre en œuvre le paradigme de service, et peut être vue comme
    une instance de l'architecture orienté service.

    Dans cette section va parler aussi d'un socle technologique très sollicité, On va aussi Détailler l'architecture de base 
    d'un service web, ensuite nous introduisons l'architecture étendus.

    \subsection{Définitions et caractéristiques}
	Les services Web sont la technologie la plus connue et la plus populaire dans le monde industriel et
	académique pour la mise en place d’architectures à services. 
	% ARE PROGRAMMABLE COMPONENTS which use the World-
	% Wide-Web as a medium for describing the functionality of real world ser-
	% vices in a computer manipulable way (Kuropka and Nern, 2006)
	Les Web services ont été proposés initialement par IBM \cite{kreger2001web} et Microsoft, puis en partie 
	standardisés par le \acrshort{w3c} \footnote{\url{http://www.w3.org/}} et définis \cite{WSA} par:\\

	\emph{``Un service web est un système conçu pour permettre d'interopérabilité des applications à travers un réseau. 
	    Il est caractérisé par un format de description interprétable/compréhensible automatiquement par la machine,
	    D'autres systèmes peuvent interagir avec le Service Web selon la manière prescrite dans sa description et en
	    utilisant des messages SOAP, généralement transmis vie le protocole HTTP et sérialisés en XML et en d'autres 
	    standards du Web ''}.\\

	Cette définition surligne les caractéristiques clés de services Web \cite{fremantle2002enterprise}:

	\SpecialItem
	\begin{description} % Les caractéristiques des services web Selon la définition de W3C
	    \item[Basés sur des protocoles Internet] : 
		L'utilisation de \acrshort{http} pour le transport des informations permet de traverser les contrôles
		d'accès dans un environnement hétérogène.

	    \item[Interopérables] : 
		Le standard \textsc{SOAP} \cite{box2000simple} définit comme étant un protocole destiné à l'échange 
		de messages structurés véhiculé généralement sur \acrshort{http} et sérialisé en \textsc{XML}, 
		permettant le support pour l'interopérabilité.

	    \item[Basés sur XML] : 
		Le méta-langage de balisage \textsc{XML} \textit{eXtensible Markup Language} est un standard Web ouvert 
		par \textsc{W3C} \cite{bray1998extensible} offre un cadre standard pour la définition de documents 
		Interprétable par des machines.
	\end{description}

	M. P. Papazoglou \cite{papazoglou2003service} apporte une autre définition de services web:\\

	\emph{``Les services Web sont des éléments auto-descriptifs et indépendants des plateformes qui permettent 
	    la composition faible coût d’applications distribuées. Les services Web effectuent des fonctions allant 
	    de simples requêtes des processus métiers complexes. Les services Web permettent aux organisations d’exposer 
	    leurs programmes résultats sur Internet (ou sur un intranet) en utilisant des langages (basés sur XML)
	    et des protocoles standardisés et de les mettre en œuvre via une interface auto-descriptive basée sur 
	    des formats standardisés et ouverts''}\\
	% TODO make a comment on this def and introduce the web services composition idea !

	 
	% TODO organize this mess
	% * Permettre l'interopéabilité des applications, indépendamment des plates-formes et des langages

	% * Permettre le couplage faible des applications (évolution indépendante) et leur coopération 
	% via des interfaces de haut niveau d'abstraction (services globaux)

	% * Permettre une coopération des applications avec un minimum d'intervention humaine
	Curbera et al. \cite{curbera2001web} de ça part propose la définition suivante :\\

	\emph{``Un service Web est une application réseau capable d'interagir par le moyen des standards et des protocoles 
	via des interfaces bien spécifiés, dans lequel est décris utilisant un langage de description fonctionnel
	standardisé''} .\\

	% Brièvement, un service Web est une entité logiciel modulaire, auto-descriptive et autonome accessible 
	% via Internet.

    \subsection{L'évolution des styles des services web}
    % Les architecture communes des services web
	% SOAP vs REST
	% les limites de la'approche SOAP 
	% pourquoi SOAP ?
	``the next section provides a short history of web services, with emphasis on the kinds of software challenges
	that web services are meant to address.''

    \subsection{L'architecture de référence et technlogies associées}
    \cite{curbera2002unraveling} \cite{gottschalk2002introduction} \cite{WSA}
    % The basic SOAP/WSDL/UDDI standards are a particular implementation of the concept of a service-
    % oriented architecture.
    % cette sous-section va détaillé SOAP et pas de WSDL et UDDI

    %TODO the main arch goes here !
	Les services Web sont construits autour de standards qui sont \acrshort{soap}, \acrshort{wsdl} et \acrshort{uddi} 
	assurant respectivement leur communication, leur description et leur découverte .

	% \renewcommand{\descriptionlabel}[1]{\hspace{1cm}\textbf{#1}}
	% \begin{description} % SOAP + WSDL + UDDI
	\subsubsection{Communication: SOAP}
		\textsc{SOAP}, développé par IBM\footnote{\url{http://www.ibm.com}} et
		Microsoft\footnote{\url{http://www.microsoft.com}} \cite{box2000simple}, est une recommandation \textsc{W3C} 
		qui le définit comme étant un protocole destiné à l'échange de messages structurés, permettant 
		d'invoquer des applications sur des réseaux distribués \cite{mitra2003soap}.

		Ce protocole \textsc{SOAP} est basé sur \textsc{XML} pour mettre en place un mécanisme valable
		d'échange des données indépendant du modèle de programmation de l'application et du système
		d’exploitation.

	      	Un message \textsc{SOAP} est un document XML constitué d'une enveloppe \textsc{SOAP} obligatoire, 
		d'un en-tête \textsc{SOAP} facultatif et d'un corps \textsc{SOAP} obligatoire:

		% TODO pick something real from the app.
		% \begin{figure}[h]
    \begin{Verbatim}[frame=single, fontsize=\scriptsize]
	<?xml version="1.0" encoding="utf-8"?>
	<soapenv:Envelope
	xmlns:soapenv="http://schemas.xmlsoap.org/soap/envelope/"
	xmlns:xsd="http://www.w3.org/2001/XMLSchema"
	xmlns:xsi="http://www.w3.org/2001/XMLSchema-instance">
	<soapenv:Body>
	<symbol xmlns="http://stock.samples">XXX</symbol>
	</soapenv:Body>
	</soapenv:Envelope>
    \end{Verbatim}
    \caption{Exemple de message SOAP}
    \label{fig:soap-message-example}
\end{figure}


		% TODO SOAP message structure
		% \begin{figure}
		%     \centering
		% \end{figure}

		\SpecialItem
		\begin{description} % SOAP components
		    \item[Enveloppe]:  
			L'élément racine du message \textsc{SOAP} , définissant le contexte du message, son
		       destinataire et son contenu, il englobe l'en-tête et le corps.

		    \item[En-tête \texttt{<Header>}]:
		       	Un mécanisme générique permettant d'ajouter des fonctions à un message
			\textsc{SOAP} d'une manière modulaire sans accord préalable entre les parties en communication. 
			Des exemples d'extension qui peuvent être implémentées comme des en-têtes sont des 
			authentifications, des transactions, des paiements

		    \item[Corps \texttt{<Body>}]: 
			Contient les informations obligatoires destinées à l'ultime destinataire du message, il sert 
			comme un container pour les informations mandataires à l'intention du récepteur du message.
			\textsc{SOAP} définit un élément pour le corps, qui est l'élément \texttt{<Fault>} (Erreur) 
			utilisé pour rapporter les erreurs.
		\end{description}

	\subsubsection{Description: WSDL}
	Le langage de description des services Web \acrshort{wsdl} \cite{chinnici2007web} est une recommandation du
	\acrshort{w3c}, maintenant dans sa deuxième version.  \textsc{WSDL} est basé sur \textsc{XML}
	pour décrire les fonctions opérationnelles de services Web. la description des\textsc{WSDL} sont 
	composées d'un interface et des implementation. L'interface est une définition abstraite et 
	réutilisable service qui peut être référencée par plusieurs implémentations. 	
	% TODO WSDL 2.0

	Le WSDL sert à décrire :
	% TODO
	\renewcommand{\labelitemi}{$\bullet$}
	\begin{itemize} % ce que WSDL offrire
	    \item le protocole de communication (SOAP RPC ou SOAP orienté message)
	    \item le format de messages requis pour communiquer avec ce service
	    \item les méthodes que le client peut invoquer
	    \item la localisation du service.
	\end{itemize}
        % TODO reference the section
	\subsubsection{Découverte: UDDI}
	\acrshort{uddi} \cite{clement2004uddi} est une standardisation pour la publication et la découverte 
	des services Web initialement conçue et spécifiée par le Consortium de standards OASIS\footnote{\url{https://www.oasis-open.org}},
	et il est le résultat d'un accord d'un ensemble d’industriels Ariba\footnote{\url{http://www.ariba.com/}}, 
	IBM, Microsoft, etc en vue de devenir le registre standard de la technologie des services Web.

	\textsc{UDDI} complète les technologies basiques de services Web en permettant de créer un \textbf{annuaire} 
	permettant de localiser sur le réseau le services web recherchés, les services référencés dans \textsc{UDDI} 
	sont accessibles par l'intermédiaire du protocole de communication \textsc{SOAP}, et la publication des 
	informations concernant les fournisseurs et les services doit être spécifiée en \textsc{XML} afin que la 
	recherche et l'utilisation soient faites de manière \textbf{dynamique} et \textbf{automatique}.

	Un \textsc{UDDI} peut appartenir à un domaine public comme internet ou tout autre réseau accessible à un nombre
       	non limité d’utilisateurs, comme il peut appartenir à un domaine restreint comme l'intranet d’une entreprise 
	ou d'un groupe d'entreprise.
	%TODO UDDI discovery and binding example

	Les données stockés dans l'UDDI sont structurées (en \textsc{XML}) et organisées en trois parties 
	connues:

	% TODO
	\begin{description} % les pages d'UDDI
	    \item[Pages blanches]:
		fournissent des descriptions générales sur les fournisseurs de services à savoir le nom de 
		l'entreprise qui fournit le service, son identificateur commercial, ses adresses, etc.

	    \item[Pages jaunes]:
		comportent des descriptions détaillées sur les fournisseurs de services catalogués dans les pages 
		blanches d'une de façon taxonomique (selon secteurs d'activités par exemple).

	    \item[Pages vertes]:
		fournissent des informations techniques sur les services Web catalogués. Ces informations incluent 
		la description du service, les adresses \textsc{URL}, du processus de son utilisation 
		et des protocoles utilisés pour son invocation.

	\end{description}
	% TODO talks about OWL-S and why UDDI approach is semantically poor

     % TODO Conclure cette subsection par la mention du problème de la découverte automatique des services Web et l'insuffisance
     % de la description syntaxique.

\section{Description des services web} 
    % introduction %TODO
    La description d’un service consiste à définir une interface exposant les opérations accomplies par le service et 
    lier chaque opération à sa réalisation. Dans cette section nous présentons les modèles de description des services

    %TODO page 66 medjahed.pdf
    \cite{medjahed2004thesis}
    	\subsection{Description syntaxique de services}
	% WSDL 1.1 , WSDL 2.0
        \subsection{Ajout de la sémantique}
	    % la description syntaxique est insuffisante.
	    A Semantic Web service is defined as an extension of Web service description through the Semantic Web annotations,
	    created in order to facilitate the automation of service interactions . Therefore, from 
	    he perspective of the functionality offered, Semantic Web services are still Web services. The only difference lays
	    in their description and the consequent benefits that follow, namely the reduction of human involvement in 
	    he performed interactions.\\

	    - insuffisance de description syntaxique des services web :(WSDL)
	     % What is semantic web? \\
	     RDF? \cite{lassila1999resource}\\
	    ``Ontologie is a representation of a shared conceptualisation of of a particular domain'' 
	     % \cite{decker2000semantic}\\

	\subsubsection{Annotations sémantiques};
	  % WSDL-S
	  % SAWSDL
	\subsubsection{Ontologies de services} 
	  % OWL-S
	  \cite{mcguinness2004owl} , \cite{mcilraith2003bringing}
	  % WSMO

\section{Découverte des services web}
   % parler de l' UDDI,le matching sémantique !!

\section{Conclusion}
    % faire un petit récapitulatif sur les technologies des services web 
    % rappeler de notre problème principale : composition des services web
