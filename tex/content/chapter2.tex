\chapter{La Composition des services web}
%% TODO Introduire la notion de la composition et le plan du chapitre

%   % Dans le chapitres précédent, nous avons étudiés la description, la
% publication,  la découverte et la sélection de services Web
% élémentaire. L'autre concept fondamentale % est la composition des
% services web.
  
%   % The other fundamental concept is web service composition which
% sometimes overlaps or will merge with the process of WS
% discovery. WS composition is a mechanism of combining two or more
% basic services into a possibly complex service. It is used to solve
% complex problems by combining available basic services. It helps to
% accelerate rapid application development and facilitate service
% reuse from developer perspective % and from user perspective it
% increases complex service consumption. As mentioned earlier a
% composite service can be regarded as a combination of services invoked
% in % a predefined order and executed as a whole and that has more
% functionality than its components. WS composition is needed because
% finding a right service provider for the request is not an easy task
% on fast growing WWW sometimes it is even % impossible. Thus WS
% composition becomes necessary and inevitable. Composing WS from
% existing ones is an effective method to fill this gap.\cite{Omer2011}

%   Dans ce chapitre, nous présentons dans un premier temps les
% définitions et les types de composition de services Web présents dans
% la littérature. Ensuite, nous étudions .....
  
%   Enfin, un ensemble de travaux proposent des approches de la
% composition dynamiques des services web sémantiques.
  

\newpage

  \section{Définition et types de composition}
  \label{sec:defin-et-caract}

  Cette section a pour but d'exposer, d'une part, quelques définitions et objectifs de la composition
  des services web proposées par la communié, et d'autre part, les différents types et mécanismes 
  de composition selon différents points de vue rencontrés dans la littérature.
  
    \subsection{Définitions}
    \label{sec:definitions}

    McIlraith \emph{et al.} \cite{mcilraith2003bringing} définissent la composition comme étant 
    \emph{``le processus de sélection, de combinaison et d'exécution de services en vue
      d'accomplir un objectif donné''}.\\
  
    Selon S. Dustdar et W. Schreiner \cite{dustdar2005survey} : \emph{`` L'infrastructure de 
      base des services Web suffit pour la mise en œuvre d'interactions simples entre un 
      client et un service Web. Si la mise en œuvre d'une application métier implique l'invocation 
      d'autres services web, il est nécessaire donc de combiner les fonctionnalités de plusieurs 
      services web. Dans ce cas, nous parlons d'une composition de services Web''}.\\

    En d'autre terme, La composition de services Web désigne une opération qui consiste 
    à construire de nouvelles applications ou services appelés services composites ou agrégats par 
    l'assemblage ou l'aggrégation de services existants nommés services atomiques ou élémentaires.

    Un service web est dit composé (composite, agrégat) lorsque son exécution implique des
    interactions avec d'autres services web afin de faire appel à leurs fonctionnalités.

    La composition spécifie quels services doivent être invoqués, dans quel ordre et sous
    quelles pré-conditions. Les services atomiques peuvent être soient des services atomiques
    soient des services composites.
    
    %% commment classifies la compositions ?
      % \cite{fluegge2006challenges} Static vs dynamique     
      
      \subsection{Procédés de coordination}
      \label{sec:proc-de-coord}      

      \subsection{Types de composition}
      \label{sec:types-de-composition}

  \section{Languages de composition}
  \label{sec:lang-de-comp}



  \section{Compostion dynamique}
  \label{sec:comp-dynam}
 
  \section{Conclusion}  
  \label{sec:conclusion}
 

%%% Local Variables: 
%%% mode: latex
%%% TeX-master: "../main"
%%% End: