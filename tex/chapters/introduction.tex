%!TEX root = ../../main.tex

% Dans le domaine des services web, il y'a plusiers méthode de la composition des 
% services web.\\
%
%     	L'approche à services est une approche relativement récente qui présente un certain nombre d'avantages pour 
% 	la réalisation des applications. Nous allons, dans un premier temps, définir l'élément-clé de l'approche 
% 	à services, c'est-à-dire le service. Ensuite, nous présenterons les différents acteurs et leurs interactions 
% 	pour cette approche. Dans une troisième partie, nous détaillerons l'architecture à; services. Nous expliquerons 
% 	aussi les besoins qui en découlent. Finalement, nous mettrons en évidence les éléments spécifiques qui nous 
%     	permettront par la suite de caractériser les différentes technologies qui mettent en œvre cette approche.\\
%
% 	Dans l'introduction de cette section on va définir,  les termes
% 	suivants: \emph{``Service, SOC et SOA ''}\\;
% 	    * C'est quoi un service?\\

\chapter{Introduction}


%%% Local Variables: 
%%% mode: latex
%%% TeX-master: "../main"
%%% End:  