\chapter{Les approches de composition dynamique des services Web sémantiques baseés sur le modèle graphe}
%% Introduction to the chapter
\newpage
\section{Préliminaires}
% \section{Le processus de matching et la découverte des services Web sémantiques}
% positionner l'approche par graphe dans les classification déjà
% abordées

%   \subsection{Mésures de similarité}
%   \label{sec:mesure-de-similarire}
%   objectif des mesures de similarité sémantique est d'évaluer la
%   proximité sémantique entre les concepts (auxquels les termes des
%   requêtes et documents sont rattachés). Et comme le matching des
%   services web revient à trouver les similarités entre les concepts
%   décrivant les propriétés des services web, nous avons jugé utile
%   d’explorer les différentes méthodes de recherche de similarité
%   utilisées dans le domaine de la recherche d’information, en
%   particulier.
%   \begin{mydef}[Similarité entre concepts]
%     Une similarité $\sigma$: C $\times$ C $\rightarrow$ [0,1] est une
%     fonction de couples de concepts vers un nombre compris entre 0 et
%     1 exprimant le degré de similarité entre deux concepts, tel que:
%     \SpecialItemi
%     \begin{itemize}
%       \item x $\in$ C, $\sigma$(x,x) = 1
%       \item x, y $\in$ C, $\sigma$(x,y) = $\sigma$(y,x)
%     \end{itemize}
%   \end{mydef}
%   Les ontologies définissent les propriétés des concepts et les
%   relations entre concepts. Une de ces relations est utilisée dans
%   notre application est \textsc{is–a} (est-un). Cette relation est utilisée
%   pour extraire des taxonomies à partir des ontologies

%   \begin{mydef}[Similarité entre concepts]
%     Soit C l'ensemble de concepts définis dans une ontologie. On dit
%     que T est une taxonomie et on écrit T(C, $\leq$) tel que c $\leq$
%     c' signifie que c \textit{is–a} c' (le concept c est subsumé par
%     le concept).
%   \end{mydef}
%   De nombreuses approches ont été proposées pour évaluer la similarité
%   sémantique entre deux concepts. Ces approches se divisent en trois
%   catégories [Sli, 2006]: les approches basées sur les arcs, les
%   approches basées sur le contenu informationnel et les approches
%   hybrides. %% see annexe (1)
% \section{Matching des services Web sémantiques }
% \label{sec:match-des-serv-1}
% Le processus de matchmaking repose sur la recherche de similarités
% entre les paramètres descriptives de ces derniers. Il existe deux
% catégories de paramètres de description des services web: les
% paramètres fonctionnels et les paramètres non-fonctionnels que nous
% décrivons tout de suite.
%   \subsection{Les paramètres fonctionnels}
%   \label{sec:params-fonc}
%   La sémantique fonctionnelle décrit les fonctionnalités qu’offre le
%   service en terme de transformation d’information dénotée par les
%   Entrés/Sorties (I/O) et de changement d’état après l’exécution du
%   service dénoté par les pré-conditions/effets (ou post-conditions)
%   (P/E).  Ils sont appelés ainsi parce qu’ils sont nécessaires pour le
%   fonctionnement du service web. Ces paramètres sont utiles dans la
%   recherche des similarités. En effet, grâce au formalisme de
%   description des services web (OWL-S), on trouve une description des
%   données requises pour l’exécution du service (Inputs) et une
%   description des résultats obtenus après exécution (Outputs). D’un
%   autre coté, un service web peut altérer l’état du monde après son
%   exécution.  L’état du monde requis pour que le service s’exécute est
%   la pré-condition et le nouvel état généré après son exécution est
%   l’effet du service sur le monde. Par exemple le service de logging à
%   un site web possède comme information d’entrées le username et le
%   password et l’information de sortie est un message de
%   confirmation. Après l’exécution, l’état du monde change de
%   not\_logged\_in à logged\_in. Prenons un autre exemple, un transfert
%   d’argent du compte A vers le compte B ne peut se faire que si le
%   compte A est solvable (les pré-conditions).  Les effets décrivent
%   les sorties (Output) et l’impact d’une exécution d’un service
%   web. Tant dis que le terme post-condition focalise souvent sur la
%   description des conditions concernant les valeurs retournées des
%   opérations du service web. Dans notre exemple ci-dessus, un impact
%   de l'exécution serait le transfert réel de l'argent du compte de
%   carte de crédit pour le compte de destination alors que la
%   post-condition décrit qu'aucun argent ne s’est perdu lors du
%   transfert [Car, 2007].  L’opération de matching consiste alors à
%   relier les paramètres des différents services (faire correspondre)
%   entre eux et cela suivant des règles de correspondance.
%   \subsection{Les paramètres non-fonctionnels}
%   \label{sec:params-non-fonc}
%   Les paramètres non fonctionnels sont des paramètres qui ne sont pas
%   reliés directement aux fonctionnalités d’un service web, mais mesure
%   la qualité avec laquelle le service délivre ses
%   fonctionnalités. Parmi ces paramètres on trouve:
%   \SpecialItem
%   \begin{description}
%   \item [Le coût]: C’est une propriété non-fonctionnelle qui est
%     pertinente pour le client qui veut utiliser le service.
%   \item [La sécurité]: La sécurité est une propriété non-fonctionnelle
%     qui est pertinente à la plupart des services web. Elle applique
%     des aspects tels que la communication et le cryptage des données.
%   \item [La qualité]: La propriété qualité des services est un
%     ensemble de propriétés différentes qui affecte tous les aspects
%     spécifiques du service, son utilisation et sa production tel que
%     le temps de réponse d'un appel à une opération, la capacité du
%     service...etc
%   \end{description}
%   Le matching des paramètres non-fonctionnels des services web est
%   tout aussi important dans la phase de sélections des services
%   candidats. Les propriétés non-fonctionnelles sont typiquement
%   représentés comme des politiques de services web exprimées par le
%   langage WS- Policy [Car, 2007].
%   Généralement ces approches de matching des politiques peuvent être
%   utilisées pour l'optimisation et l’évaluation des plans de
%   composition.  Les services web correspondent à des entités
%   dynamiques. Ils peuvent devenir disponibles à tout moment de
%   l’exécution du système et découverts dynamiquement par leurs
%   clients. Les fournisseurs et les consommateurs de services doivent
%   donc disposer d’un moyen commun et fiable pour effectuer la
%   publication et la recherche de services. On entend par découverte
%   dynamique la possibilité de localiser automatiquement un ensemble de
%   services web, sélectionner un service spécifique qui répond à des
%   besoins particuliers de l’utilisateur.  Plusieurs initiatives ont
%   traité la problématique de la découverte des services web dont les
%   deux principales modèles suivants:
%   \subsection{Le modèle syntaxique de matching des services Web}
%   \label{sec:le-modele-syntaxique}
%   \subsection{Le modèle sémantique de matching des services Web}
%   \label{sec:le-modele-semantique}
%   \subsection{Graphe de dépendance}
%   \label{sec:graph-de-depandence}
%   \cite{Omer2011}

% TODO
% requirement for good web service composition

\newpage
\section{Travaux connexes}
\label{sec:travaux-relatives}
% TODO: une petite introduction aux travaux connexes

% La découverte d'un service composite à partir d'un graphe de
% dépendance consiste à trouver le meilleur chemin dans le graphe qui
% génère les sorties exprimées dans la requête. La recherche peut se
% baser sur l'optimisation d’une fonction d'utilité qui tient compte du
% degré de matching entre les services et aussi de la qualité de service
% qui peut être fournie par les services concrets. A cet effet, les
% méthodes de recherche du meilleur chemin de la théorie des graphes
% peuvent être exploitées.  Les méthodes de composition utilisant le
% modèle de graphe peuvent être classées en deux catégories selon que le
% graphe de dépendance soit construit à priori (en phase de publication
% des services) ou pendant le traitement de la requête de composition:

% Nacera (TODO: améliorer cette introduction)
% Certaines approches ont été proposées récemment pour la découverte
% dynamique d'un service composite à partir d'un graphe de service. En
% effet un graphe de service est généralement mis en place pour
% représenter les dépendances possibles entre les services en terme de
% flux de données (Entrées/Sorties). Cette dépendance est déterminée
% grâce à un mécanisme de matching (mise en correspondance) entre les
% services. Cette représentation permet d'exploiter certaines
% techniques en théorie de graphe pour la recherche du meilleur plan de
% composition.

% La découverte d'un service composite à partir d'un graphe de
% dépendance consiste à trouver le meilleur chemin dans le graphe qui
% génère les sorties exprimées dans la requête. La recherche peut se
% baser sur l'optimisation d'une fonction d'utilité qui tient compte du
% degré de matching entre les services et aussi de la qualité de service
% qui peut être fournie par les services concrets. A cet effet, les
% méthodes de recherche du meilleur chemin de la théorie des graphes
% peuvent être exploitées.  Les méthodes de composition utilisant le
% modèle de graphe peuvent être classées en deux catégories selon que le
% graphe de dépendance soit construit à priori (en phase de publication
% des services) ou pendant le traitement de la requête de composition

  \subsection{Génération Online du graphe de dépendance}
  \label{sec:generation-online}

  Dans \cite{liang2005and}, Liang \textit{et al.}  Présentent une
  formalisation du problème de composition de service comme un
  problème de recherche dans un graphe \textit{And/Or}. Donc un graphe
  de dépendance (SDG) est construit dynamiquement suite à une requête
  dans un domaine spécifique choisie par l'utilisateur. Un graphe
  \textit{And/Or} contient des nœuds et des connecteurs de deux types:
  connecteur And/connecteur Or.

  Un connecteur And reliant des services avec un autre service s'il
  existe une relation de type and entre les paramètres de ces
  services, càd, toutes les entrées fournis par les services doivent
  être disponibles pour l'exécution du service destinataire. Des
  services sont connectés par un Or avec un autre service s'il existe
  une relation logique de type Or entre les paramètres des ces
  services, càd, n'importe quel service pourra produire l'entrée
  requise par le service destinataire. C'est la caractéristique
  principale de cette approche: tenir compte des deux types de
  branchement \textit{And/Or}.  Après, un algorithme de recherche
  itératif est appliqué pour trouver le service composite minimal et
  complet satisfaisant la requête puis soumis à évaluation par
  l'utilisateur jusqu'à ce qu'il valide le résultat (méthode
  semi-automatique).

  % In 2005, Altheya Lang and Y.W.Su [5] presented a model for web
  % service composition based on AND/OR graph, and a graph search
  % algorithm for searching the graph to find out the composite
  % service(s) that satisfies a user request. For a given service
  % request that only can be fulfilled by a composition of web
  % services, their algorithm find the service categories that are
  % relevant to the request and dynamically create an AND/OR graph
  % to grasp the functional dependencies among the web services of
  % these service categories. The graph is changed based on the
  % information reflected in a request. The search algorithm is used
  % to search the changed AND/OR graph for a minimal and adequate
  % composite service template that satisfies the service demand.
  % The algorithm can be executed repeatedly on the graph to find
  % out different templates until the result is considered by the
  % service requester

  Dans \cite{lecue2006formal}, Freddy et Léger considèrent également
  l'existence d'un mécanisme de découverte des services candidats,
  puis établir toutes les relations de dépendance entre ces services
  en utilisant la notion de lien causal et sauvegarde des ces
  informations dans une matrice d'adjacence.

  % Optimizing qos-aware semantic web service composition
  % \cite{lecue2009optimizing}

  % \cite{Omer2011}

  Dans \cite{omer2009dependency}, les auteurs supposent que les
  services candidates avaient été déjà découverts localement par un
  algorithme de découverte orienté but. L'exécution du plan se déroule
  en quatre principales étapes.

  \SpecialItem
  \begin{itemize}
  \item La première est la génération du graphe de dépendance grâce à
    un algorithme qui utilise une matrice d'adjacence des dépendances
    Inputs/Outputs entre les services. Les dépendances sont identifiés
    si au moins une paire de paramètre (des Input(WS1) et Output(WS2))
    sont en relation exact ou plug-in selon [Paolucci], càd, In(WS1)
    Out (WS2).

  \item La deuxième étape consiste à extraire du graphe les
    dépendances cyclique (de type Loop) en utilisant l'algorithme [4]
    puis régénérer un nouveau graphe de dépendance acyclique en
    remplaçant chaque sous- graphe cyclique par un nœud composé

  \item La troisième étape est l'extraction du plan de
    composition. Pour générer enfin le plan final, les nœuds composés
    seront remplacés par leurs sous-plans respectifs.
  \end{itemize}

  La principale contribution de ce travail est la manière de traiter
  les cas de boucle dans un graphe, ce problème particulièrement a été
  très peu abordé dans ce domaine.

  \subsection{Génération Offline du graphe de dépendance}
  \label{sec:gener-offline}
  certains auteurs considèrent qu'il serait plus intéressant
  d'utiliser des informations complètes sur tous les services publiés
  pour assurer une recherche plus exhaustive des services candidats à
  la composition. Pour cela, le graphe de dépendance est pré-généré à
  partir du registre ou sont publiés les services. Par conséquent, le
  graphe inclut toutes les dépendances possibles entre les services
  disponibles, par contre il peut être volumineux et doit être mis à
  jour au fur et à mesure des changements (l'arrivée ou non
  disponibilité des services).

  Dans \cite{hashemian2006graph} le graphe de depdendance est
  construit à priori dont les sommets représentent les paramètres
  (Inputs/outputs) et les arcs représentent les services (utilisant
  ces paramètres). Suite à une requête, le processus de composition se
  déroulera en deux étapes : a) trouver les services potentiels
  pouvant participer dans la composition en cherchant les sous graphes
  couvrant tous Ventrées et Vsorties (fournies dans la requête) en
  utilisant une recherche en largeur d'abords (breadth first
  search). b) La deuxième étape consiste à extraire les plans
  d'exécution en utilisant à partir des services découverts. Les
  structures de composition considérées dans ce travail sont les
  séquences, les branchements conditionnels.

  Dans \cite{elmaghraoui2011graph}, les auteurs modélisent à priori
  toutes les dépendances entre les services et aussi sauvegarde tous
  les meilleurs chemins entre les nœuds du graphe dans une matrice en
  utilisant l'algorithme de Floyed, en optimisant le degré de
  matching, temps d'exécution, coût et la disponibilité du service.

  \subsection{Autres travaux basés sur le graphe matching}
  \label{sec:autres-travaux}
  Alfredo Cuzzocrea, Marco Fisichella: ‘A Flexible Graph-based
  Approach for Matching Composite Semantic Web Services’, 2011:
  modélise le service composite décrit dans OWL-S Process-Model sous
  forme d’un graphe orienté (modélise le flux de données+structure de
  contrôle) puis il fait le matching à deux niveaux avec la requête de
  l’utilisateur (suppose aussi que la requête fournit également le
  schéma de composition):

  - Matching des Inputs/Outputs de la requête
  à celle du service composite

  - Matching de la structure du service décrit dans la requête avec
  celle du service composite publié (la requête est aussi décrite sous
  forme de graphe) matching de pattern

  \subsection{Discussion et comparaison}
  \label{sec:discussion-comparaison}
  % comparaison
  \begin{table}[htb!]
  \centering
  \resizebox{1.05\textwidth}{!}{%
    \begin{tabular}{|>{\centering\arraybackslash}m{1.5in}|>{\centering\arraybackslash}m{0.9in}|
        >{\centering\arraybackslash}m{1.6in}|>{\centering\arraybackslash}m{1in}|>{\centering\arraybackslash}m{1.4in}|@{}m{0pt}@{}}
      \hline
      \textbf{Approche}  & \textbf{Génération du graph} & \textbf{Algorithme utilsé}& \textbf{Sémantique} &  \textbf{Les paramètres non-fonctionnels} &\\
      \hline\hline %--------------------------------------------------------------------------------------------------
      % ---------------------------------------------------------------------------------------------------------------
      Liang \textit{et al.} \cite{liang2005and} & Online & Recherche dans un And/Or graph & Non & Non &\\ [6ex] %done
      \hline %\item ------------------------------------------------------------------------------------------------------
      Arpinar \textit{et al.} \cite{arpinar2005ontology}& Offline&Bellman-Ford& Oui & temps d'exécution &\\ [6ex] %done
      \hline %--------------------------------------------------------------------------------------------------------
      Freddy et Léger \cite{lecue2006formal} & Online &  &  &  &\\  [6ex]
      \hline %--------------------------------------------------------------------------------------------------------
      Hashemian \textit{et al.} \cite{hashemian2006graph} & Offline & Recherche en largeur d'abord (BFS) ou en profondeur (DFS) & Oui & Non&\\[12ex] %done
      \hline %-------------------------------------------------------------------------------------------------------
      Gu Zhifeng \textit{et al.} \cite{gu2008automatic}& Offline & Extension de And/Or graph utilisée par \cite{liang2005and}  & Non & Non &\\[12ex]
      \hline %--------------------------------------------------------------------------------------------------------
      Abrehet et Schill \cite{omer2009dependency}& Online & Variation d'une recherche topologique basée sur \cite{ma2007systematic}& Oui & Non &\\[12ex] %done
      \hline %--------------------------------------------------------------------------------------------------------
      Elmaghraoui \textit{et al.} \cite{elmaghraoui2011graph}& Offline & Extension de Floyd-Warshall & Oui & Coût, temps d'exécution, disponiblité.&\\[12ex] % done
      \hline %--------------------------------------------------------------------------------------------------------
      Samuel et Sasipraba \textit{et al.} \cite{samuel2011approach}& TODO & TODO & TODO & TODO&\\[10ex]
      \hline %--------------------------------------------------------------------------------------------------------
      Chaker Ben Mahmoud \textit{et al.} \cite{mahmoud2013towards} &Online & Non définie & Oui & Non &\\[6ex] %done
      \hline %--------------------------------------------------------------------------------------------------------
    \end{tabular}}
  \newline
  \caption{Comparaison des approches de composition basées sur le modèle graph}
  \label{comparaison-graph-composition}
\end{table}
%%% Local Variables:
%%% mode: latex
%%% TeX-master: "../main"
%%% End:

  % TODO
\section{Vers les bases de données graphe}
\label{sec:vers-les-bases}

\section{Conclusion}
\label{sec:conclusion-2}

%%% Local Variables:
%%% mode: latex
%%% TeX-master: "../main"
%%% End: